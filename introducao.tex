\section{Introdução}
\label{sec:introduction}

Que presidentes frequentemente recorrem à formação de coalizões para governar não constitui mais novidade alguma. Nos últimos anos, o que a literatura sobre o presidencialismo vem procurando explicar é, justamente, como isso ocorre. A tese básica que emergiu a partir de então é a de que essas coalizões são administradas através de um conjunto de ferramentas, uma \textit{toolbox}, que pode ser estrategicamente empregada pelo presidente para incentivar a cooperação e coordenar a ação dos partidos no Congresso (CHAISTY et. al., \citeyear{chaisty2014}). Dentre outros, poder de agenda (uso de decretos legislativos e pedidos de urgência, etc.), controle sobre a execução e formulação do orçamento (\textit{pork}) e distribuição de ministérios (\textit{coalition goods}) seriam os principais instrumentos que permitiriam a superação dos "perigos"{} inerentes aos regimes com separação de poderes. Como o uso dessas ferramentas varia entre países e contextos, entretanto, é algo que levanta uma série de questões.

No caso específico da distribuição de ministérios, por exemplo, os presidentes devem decidir qual será o tamanho da coalizão e quão bem recompensados serão os seus parceiros. Por sua vez, essa decisão é, ela própria, parcialmente determinada por outros fatores. Sobre isto, a literatura não vai muito além de destacar que a existência de decretos legislativos e vetos parciais podem incentivar um presidente a governar de forma imperial, distribuindo poucos ministérios (AMORIM NETO,  \citeyear{neto2000}, \citeyear{neto2006}; COX e MORGENSTERN, \citeyear{cox2001}); e que o número mínimo de partidos que o presidente precisa incorporar também depende da posição que ele ocupa no espectro ideológico e do número de cadeiras que seu partido dispõe (CHEIBUB, \citeyear{cheibub2007}; FIGUEIREDO et al., \citeyear{figueiredo2012}; NEGRETTO, \citeyear{negretto2006}). Contudo, isso não explica como essas variáveis interagem com outras ferramentas da \textit{toolbox}\footnote{Ao analisar o \textit{trade-off} entre incentivos seletivos e distribuição de ministérios no gerenciamento das coalizões dos governos FHC e Lula, Raile, Pereira e Power (\citeyear{raile2010}) constituem exceção, embora não abordem o principal problema desenvolvido no restante deste artigo.}, tampouco nos fornecem uma explicação completa para a variação no tamanho das coalizões governamentais em sistemas presidencialistas.

Isso é ainda mais sintomático quando se procura entender o surgimento de coalizões sobredimensionadas, nas quais existem mais partidos do que o necessário para se obter maioria absoluta no congresso. Em primeiro lugar, porque, como Riker (\citeyear{riker1962}) demonstrou, mais membros numa coalizão significa menos cargos à disposição de cada membro do governo. À luz disso, formar coalizões sobredimensionadas seria um erro. Segundo, porque em coalizões fragmentadas e heterogêneas as políticas ideais de cada partido dificilmente podem ser realizadas simultaneamente, o que com frequência gera custos de transação e conflitos entre estes (AXELROD, \citeyear{axelrod1970}). E, terceiro, porque delegar ministérios com jurisdições específicas pode incentivar cada ministro a buscar políticas que sirvam apenas ao seu partido, mesmo que isso prejudique os demais parceiros da coalizão (SAALFELD, \citeyear{saalfeld2000}; MARTIN e VANBERG, \citeyear{martin2011}; FREITAS, \citeyear{freitas2012})\footnote{Uma leva de estudos argumenta que estes problemas de \textit{moral hazard} decorrentes dessa relação agente-principal são contornáveis (e. g., AMORIM e TAFNER, \citeyear{neto2002}; FREITAS, \citeyear{freitas2012}; STR\O{}M et al., \citeyear{strom2010}). Mesmo assim, essas correções trazem custos: esse é ponto principal do argumento aqui utilizado.}. Considerando a frequência com que ocorrem - Figueiredo et al. (\citeyear{figueiredo2012}, p. 847) reportam que mais de 35\% dos governos na América Latina entre 1979 e 2011 foram supermajoritários -, portanto, cabe perguntar: por que presidentes propõem, e partidos aceitam integrar, coalizões sobredimensionadas? Por que aqueles não cedem apenas um número suficiente de ministérios para obter maioria? E, em todo o caso, os mesmos fatores que explicam o surgimento de coalizões menores também explicam as maiores? E qual é a relação entre coalizões grandes e demais ferramentas da \textit{toolbox}? Apesar da importâncias dessas questões, as respostas na literatura a essas perguntas são vagas e estudos comparados, inexistentes.
 
Neste artigo, procuro exatamente preencher esta lacuna. Com um banco de dados que cobre todos os 18 países presidencialistas da América Latina após a terceira onda da democratização, testo algumas daquelas respostas presentes na literatura com análise multivariada. Ao invés da mais recorrente, a de que, antecipando a indisciplina dos partidos que o sustentam, os presidentes aumentariam o tamanho de suas coalizões, sugiro uma outra, relacionada às necessidades de gerenciamento da coalizão: a de que quanto mais forte for o poder legislativo, mais partidos na coalizão serão necessários para formar um cartel legislativo. Um exemplo disso ocorre no governo Dilma I, enquanto escrevo esta introdução. 

Insatisfeitos com o governo e com a política de alianças regionais do PT, alguns parlamentares lançaram ofensivas que, no plenário e na Comissão de Agricultura, Pecuária, Abastecimento e Desenvolvimento Rural da Câmara, respectivamente, derrubou um decreto da Presidente e convocou dois Ministros\footnote{O ESTADO DE SÃO PAULO. "Câmara derruba decreto de conselho popular de Dilma". Disponível em: \url{http://goo.gl/rJPuXB}. Acesso em: 3 de novembro de 2014.} \footnote{CONGRESSO EM FOCO. "Oposição supera bloqueio na Câmara e convoca dois ministros". Disponível em:~\url{http://goo.gl/W7ZNEY}. Acesso em: 3 de novembro de 2014}. Formalmente, essas ações foram possíveis porque ao Presidente da Câmara cabe estabelecer a agenda de votações, e a uma maioria simples dos membros de cada Comissões aprovar pedidos para depoimentos. Atualmente, esses espaços são controlados pela base governista. Sendo assim, qualquer ação contra o governo, nestes espaços, só ocorrem se contarem com o suporte de membros da coalizão\footnote{Sobre isto, a nota do Congresso em Foco é ilustrativa: "\textit{Da base aliada, poucos deputados se manifestaram contra a convocação dos ministros}. Márcio Macedo (PT-SE) sugeriu que os pedidos fossem transformados em convite.[...] Porém, como a maioria da comissão faz parte da bancada ruralista, atualmente em pé de guerra com o Planalto, as tentativas dos governistas em modificar os requerimentos acabou fracassada." (Ênfase minha).}. Em qualquer outro caso, a oposição não prevalece.

Independentemente do desfecho, ambos os exemplos ilustram perfeitamente como o sistema de comissões e a organização do legislativo podem oferecer amplas prerrogativas de veto e revisão legislativa, controle da agenda e mesmo investigação de outros poderes. Quanto mais numerosos e efetivos forem esses espaços, mais parlamentares serão necessários para os controlar. Em outras palavras, coalizões governamentais, quando funcionam, também servem para obter a cooperação de legislativos fortes o suficiente para contrariar os interesses de um presidente.

No restante do artigo, procedo da seguinte forma. Após discutir brevemente os principais parâmetros nos modelos sobre formação de coalizões governamentais no parlamentarismo, discuto algumas adaptações destes para o presidencialismo e os problemas que geraram na seção~\hyperref[sec:revisao]{2}. Na seção~\hyperref[sec:FAZER]{3}, introduzo as hipóteses a serem testadas sobre os determinantes das coalizões sobredimensionadas na Amérila Latina. Na seção~\hyperref[sec:FAZER]{4}, discuto o método e apresento os dados utilizados, que cobrem o período entre 1978 a 2010 e incluem todos os 18 países presidencialistas da América Latina -- incluso os da América Central, frequentemente ignorados pela literatura comparada. Nas seções~\hyperref[sec:FAZER]{5} e~\hyperref[sec:FAZER]{6}, apresento e discuto os resultados. Por fim, seguem as conclusões.
