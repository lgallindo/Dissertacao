\subsection{Governos de coalizão no presidencialismo}

Em sistemas presidencialistas, coalizões governamentais supostamente não deveriam surgir. Dois argumentos principais sustentam essa conclusão. O primeiro é o de que presidentes não dependeriam do legislativo para manter seus cargos e, legitimados pela maioria dos eleitores, poderiam governar sem o apoio de uma maioria. Essa é a essência da dinâmica \textit{winner-takes-all} inerente ao presidencialismo (LINZ, \citeyear{linz1990}). O segundo argumento, desenvolvido por uma outra geração de comparativistas, é o de que certos aspectos institucionais comuns ao presidencialismo dificultariam o surgimento de coalizões. Representação proporcional para eleições legislativas tornaria bastante provável que o partido do presidente não contasse com apoio legislativo suficiente para aprovar leis. Sistemas eleitorais inclusivos, por outro lado, gerariam incentivos para a personalização e regionalização das campanhas eleitorais, enfraquecendo os partidos políticos e tornando difícil negociar coalizões (MAINWARING, \citeyear{mainwaring1993}; SHUGART e CAREY, \citeyear{shugart1992}; STEPAN e SKACH, \citeyear{stepan1993}).

O que ambos os argumentos põem em causa são, precisamente, os incentivos que poderiam levar à cooperação. Pelo lado dos presidentes, incorporar membros de outros partidos em seus ministérios não seria vantajoso. Quando necessário, ferramentas como decretos legislativos e pedidos de urgência os permitiriam governar por cima do congresso, no máximo usando do poder discricionário sobre cargos e orçamento para formar coalizões legislativas \textit{ad hoc}. Do ponto de vista dos parlamentares, as campanhas individualizadas de países como Brasil, Colômbia e Peru os incentivariam ao paroquialismo, à distribuição de \textit{pork} e à patronagem. Além disso, o federalismo e a tibieza organizativa dos partidos, comuns na América Latina, tornariam problemática a coordenação intrapartidária. Assim, tratar os partidos como atores unitários, algo trivial naqueles modelos sobre formação de coalizões no parlamentarismo, seria problemático para o caso do presidencialismo.

Nos últimos anos, uma série de estudos vem reavaliando essas proposições, especialmente pela falta de corroboração empírica a elas: governos de coalizão no presidencialismo são tão frequentes quanto no parlamentarismo, e hoje constituem a norma em países como Brasil, Uruguai, Chile e Colômbia (ALTMAN, \citeyear{altman2000}; CHEIBUB, \citeyear{cheibub2007}; CHEIBUB et al., \citeyear{cheibub2004}). Diante deste "sucesso inesperado", o problema é o de, a despeito dos incentivos institucionais que levariam ao conflito, explicar como a cooperação surge e se mantém (AMORIM NETO, \citeyear{neto2006}; CHAISTY et al., \citeyear{chaisty2014}; HIROI e RENNÓ, \citeyear{hiroi2014}; MELO e PEREIRA, \citeyear{melo2013}). Sobre isso, alguns estudos recentes destacam alguns pontos.

Em primeiro lugar, presidentes podem cooperar quando os custos de estar em conflito com o congresso forem muito altos. Um dos aspectos em que isso é mais palpável é no uso de decretos legislativos. Por exemplo, Negretto (\citeyear{negretto2004}) sugere que, para usá-los de forma exclusivamente unilateral, é necessário, entre outros, controlar o membro pivotal do congresso, isto é, contar com um base suficiente para impedir a aprovação de vetos, como foi o caso dos Presidentes Alfonsín e Menem na Argentina. Na ausência desta condição, expedir decretos pode apenas funcionar como uma ferramenta de coordenação e delegação, transferindo os custos da tomada de decisões do congresso para o executivo (Cf. CAREY e SHUGART, \citeyear{carey1998})\footnote{Reconhecidamente, ao menos para o Brasil, não existe evidência que suportem fortemente ou a teoria unilateral ou a delegativa sobre o uso de decretos legislativos (PEREIRA et al., \citeyear{pereira2005})}. Teoricamente, apenas no primeiro caso é que presidentes minoritários poderiam dispensar a formação de coalizões governamentais (FIGUEIREDO et al., \citeyear{figueiredo2012}; NEGRETTO, \citeyear{negretto2006}).

Outra coisa que o congresso também pode fazer pelo presidente é fornecer suporte em momentos de crise. Como a interrupção de mandatos não é incomum no continente, presidentes que buscam governar de forma imperial, contrariando os interesses do congresso, estariam mais suscetíveis a cair quando, e. g., crises econômicas ou protestos ocorrem (ÁLVAREZ e MARSTEINTREDET, \citeyear{alvarez2010}; HOCHSTETLER, \citeyear{hochstetler2006}; NEGRETTO, \citeyear{negretto2006}). Em situações de estabilidade democrática, portanto, haveria uma tendência de preponderância do legislativo: este pode impor consideráveis \textit{checks} ao executivo e, mesmo assim, não pode ser dissolvido por este sem que o ordenamento jurídico seja interrompido (PÉREZ-LIÑÁN, \citeyear{perez2005})\footnote{Na Colômbia e na Venezuela, assembleias constituintes foram convocadas e, durante seus trabalhos, dissolveram o congresso. Ainda que não tenha sido por intervenção militar, como o fracassado auto-golpe de Fujimori, no Peru, estas dissoluções não eram diretamente previstas pelas respectivas constituições -- como a do Uruguai, que prevê tal prerrogativa, apesar de que, na prática, ela não é usada (SHUGART e CAREY, \citeyear{shugart1992}, p. XX)}. Assim, contar com apoio no congresso seria vantajoso não só por facilitar a aprovação da agenda do presidente e conferir legitimidade a ela (Cf. AMORIM NETO, \citeyear{neto2006}), mas também porque permitiria aos presidentes terem governos mais estáveis e maiores chances de concluírem seus mandatos. Do contrário, haveria sempre a possibilidade de que o congresso governasse por cima do presidente\footnote{Há, também, a mesma questão para o legislativo: por que partidos no congresso cooperariam com o presidente? Duas respostas aqui são possíveis. A primeira é de que, se eles maximizam cargos, participar de um ministério é algo vantajoso. A segunda é a de que, se maximizam \textit{office} e \textit{policy}, os partidos comparariam a utilidade de estar fora do governo vis-à-vis a chance de vencer as eleições com a utilidade de participar do governo (CHEIBUB, \citeyear{cheibub2007}). Assim, a questão principal não é tanto da oferta de parceiros de coalizão, mas de demanda por cooperação.}.

Apesar de tudo isso, de nada adiantaria aos presidentes cooperar com o congresso se os partidos não possuíssem a capacidade de fornecer apoio de forma coordenada. Nesse aspecto, também, a literatura passou a destacar que a influência de sistemas eleitorais mais permissivos, com lista aberta e magnitudes elevadas, podem ser neutralizados na arena legislativa através do controle da agenda legislativa e da centralização decisória (CHEIBUB, \citeyear{cheibub2007}; CITAR). Do mesmo modo, a ausência de voto de confiança e de lista fechada não são indicativos necessários de que um país possui altos níveis de indisciplina partidária ou, mesmo que possua, que isso acarrete em \textit{gridlock}. Como Cheibub (\citeyear{cheibub2007}, p. 132) argumenta, pode ser do interesse dos líderes dispensar parte de suas bancadas em certas votações, especialmente naqueles onde uma maioria já está garantida; fazendo isso, eles tornam mais simples a tarefa de conciliar demandas particularistas dos deputados com as coletivas, do partido\footnote{Há também a possibilidade de que, se os parlamentares fossem indisciplinados, os presidentes poderiam costurar maiorias \textit{ad hoc} e praticar \textit{retail politics} (AMORIM NETO, \citeyear{neto2006}, p. XX). Curiosamente, um dos defensores desta proposição é Juan Linz (\citeyear{linz1990}, p. XX), num trecho frequentemente ignorado de um de seus artigos mais famosos. Para ele, essa seria, provalmente, a fórmula da estabilidade do presidencialismo nos Estados Unidos. DESENVOLVER MAIS? EXISTEM ESTUDOS EMPÍRICOS QUE CORROBORAM ESSE PONTO.}.

De forma semelhante, \textit{surveys} com parlamentares e \textit{experts} no continente afirmam que a percepção do posicionamento ideológico dos partidos latinoamericanos é razoavelmente estruturada\footnote{Ainda que apresentem problemas de subjetividade (a escala varia entre indivíduos) e erros de mensuração, modelos de variável latente permitem corrigir estes, como o utilizado por Zucco (ANO)}. Adicionalmente, outros estudos se valeram da escala desenvolvida por Coppdge (ANO), que classifica os partidos num \textit{continuum} de 0 a 5 na esquerda e direita, em termos socioeconômicos, e obtiveram resultados significativos e com sinais condizentes com as hipóteses testadas. Destes, Aleman e Tsebelis (\citeyear{aleman2011}) encontraram evidências de que partidos ideologicamente mais próximos do partido do presidente possuem maiores chances de vir a integrar a coalizão governista; e Martinez-Gallardo (\citeyear{martinez2012}), por sua vez, mostra que coalizões mais homogêneas tendem a ser mais estáveis. Alguma evidência em favor da importância dos partidos como estruturadores dos governos de coalizão, portanto, já existe.

Cheibub et al. (\citeyear{cheibub2004}) e Cheibub (\citeyear{cheibub2007}) levam essa possibilidade adiante com um modelo formal, adaptado de Austen-Smith e Banks (\citeyear{austen1988}), no qual o partido do presidente é sempre o \textit{formateur}. A principal conclusão deles, contudo, é semelhante: governos minoritários podem existir no presidencialismo pela mesma razão que existem e se mantêm no parlamentarismo, qual seja, ausência de uma coalizão alternativa com incentivos para governar por cima do presidente. Figueiredo et al. (\citeyear{figueiredo2012}) também apresentam evidências de que presidentes minoritários estão associados com poderes legislativos fortes, como o de expedir decretos, e alta fragmentação partidária. Raile et al. (\citeyear{raile2010}) mostram que há, dentro de certos limites, um \textit{trade-off} entre emendas orçamentarias e distribuição de ministérios no Brasil, o que tem certo impacto no tamanho das coalizões. Ainda outros estudos se debruçam sobre os determinantes da proporcionalidade da distribuição de ministérios (AMORIM NETO, \citeyear{neto2006}); sobre o perfil dos ministros (CAMERLO e PÉREZ-LIÑÁN, \citeyear{camerlo2012}); e sobre a influência do tamanho das coalizões governamentais no desenvolvimento de agências regulatórias autônomas (MELO e PEREIRA, \citeyear{melo2013}), no uso de obstrução legislativa (HIROI e RENNÓ, \citeyear{hiroi2014}) e na aprovação de reformas estruturais (ALTMAN e CASTIGLIONI, \citeyear{altman2008}). Mas e o que explica o tamanho dessas coalizões, em primeiro lugar? FALTA UM PARAGRAFO FINAL.
