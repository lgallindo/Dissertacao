\section{Tamanho das coalizões no presidencialismo}

Ao contrário da literatura sobre o parlamentarismo, os estudos sobre sistemas presidencialistas até pouco tempo afirmavam que coalizões governamentais não deveriam surgir. Dois argumentos principais sustentariam essa conclusão. O primeiro é o de que, legitimados pela maioria dos eleitores, presidentes não dependeriam do apoio do legislativo para manter seus cargos. Essa seria a essência da dinâmica \textit{winner-takes-all} inerente ao presidencialismo (LINZ, \citeyear{linz1990}; LINZ e VALENZUELA, \citeyear{linz1994}; RIGGS, \citeyear{riggs1988}). O segundo argumento, desenvolvido por outra geração de comparativistas, é o de que certos aspectos institucionais comuns ao presidencialismo latino-americano dificultariam o surgimento de coalizões (MAINWARING, \citeyear{mainwaring1993}; SHUGART e CAREY, \citeyear{shugart1992}; STEPAN e SKACH, \citeyear{stepan1993}). Representação proporcional para eleições legislativas tornaria bastante provável que o partido do presidente não contasse com apoio legislativo de uma maioria. Sistemas eleitorais inclusivos, por outro lado, gerariam incentivos para a personalização e a regionalização das campanhas eleitorais, o que em último caso enfraqueceria a coesão dos partidos políticos e dificultaria a formação e a manutenção de coalizões. Deste modo, os presidentes da região não teriam incentivos para cooperar com as assembleias, preferindo antes contorná-las ou usar incentivos seletivos para formar coalizões legislativas \textit{ad hoc} (COX e MORGENSTERN, \citeyear{cox2001}; JONES, \citeyear{jones1995})\footnote{Reconstituir este debate está fora do escopo deste artigo. Boas revisões, entretanto, podem ser encontradas em Cheibub (\citeyear{cheibub2007}), Power (\citeyear{power2010}) e Negretto (\citeyear{negretto2006}).}.

Como essa premissa de ausência de cooperação entre executivo e legislativo colocava em causa a própria estabilidade democrática e a governabilidade dos países da América Latina, os primeiros estudos sobre o presidencialismo buscaram invariavelmente explorar as condições favoráveis à formação de governos multipartidários ao invés das diferenças entre estes. Passados alguns anos do início deste debate, hoje sabemos que coalizões governamentais são comuns e que auxiliam na aprovação da agenda presidencial, na diminuição de conflitos intergovernamentais e na sustentação de presidentes que enfrentam crises econômicas ou protestos populares (ÁLVAREZ e MARSTEINTREDET, \citeyear{alvarez2010}; CHEIBUB, \citeyear{cheibub2007}; CHEIBUB et al., \citeyear{cheibub2004}; HOCHSTETLER, \citeyear{hochstetler2006}; NEGRETTO, \citeyear{negretto2006}; PÉREZ-LIÑÁN, \citeyear{perez2007})\footnote{Na Colômbia e na Venezuela, assembleias constituintes foram convocadas e, durante seus trabalhos, dissolveram o congresso. Ainda que não tenha sido por intervenção militar, como o fracassado auto-golpe de Fujimori, no Peru, estas dissoluções não eram diretamente previstas pelas respectivas constituições –- ao contrário da uruguaia, que prevê tal prerrogativa, apesar de que, na prática, ela não seja usada (SHUGART e CAREY, \citeyear{shugart1992}, p. 127)}. Recentemente, também surgiram estudos mostrando que coalizões de diferentes tamanhos trazem consequências distintas, como promover a autonomia de agências regulatórias quando são estabelecidas, aumentar o grau de obstrução legislativa à agenda do presidente e facilitar a implementação de reformas estruturais (ALTMAN e CASTIGLIONI, \citeyear{altman2008}; HIROI e RENNÓ, \citeyear{hiroi2014}; MELO e PEREIRA, \citeyear{melo2013}). Mas, apesar desses avanços, à exceção de uns poucos estudos que analisam a composição e a estabilidade de gabinetes presidencialistas comparativamente (e. g., AMORIM NETO, \citeyear{neto2006}; FIGUEIREDO et al., \citeyear{figueiredo2012}; MARTINEZ-GALLARDO, \citeyear{martinez2012}), ainda sabemos pouco sobre o que explica essas diferenças no tamanho das coalizões que são formadas no Continente em primeiro lugar.

Em parte, o tamanho das coalizões poderia ser considerado função exclusiva da decisão dos presidentes, já que eles possuem a prerrogativa de nomear os ministros no presidencialismo. Contudo, esse dificilmente é o caso. A montagem de um gabinete multipartidário envolve a interação estratégica de diversos atores com preferências, poder de barganha e prerrogativas institucionais diferentes. Em decorrência disso, segundo Cheibub (\citeyear{cheibub2007}), a formação de uma coalizão é um jogo no qual os presidentes fazem propostas para os potenciais parceiros considerando os custos e a utilidade de tê-los cooperando e a probabilidade de eles aceitarem essas propostas. Se todos os jogadores procurarem implementar políticas públicas e obter cargos, coalizões surgiriam exceto quando as preferências ideológicas do presidente fossem tão extremas que nenhum partido obtivesse vantagem ao integrar o gabinete; ou quando o presidente estivesse centralmente localizado no espectro ideológico e fosse naturalmente o ponto de convergência da maioria. Em outras palavras, ainda que sejam eleitos separadamente e muitas vezes contem com amplos poderes legislativos, presidentes não escolhem arbitrariamente suas coalizões, já que as preferências dos demais partidos e o contexto no qual as negociações ocorrem produzem incentivos que determinam parcialmente os resultados do processo.

Da evidência disponível sobre governos de coalizão no presidencialismo, a maioria é consistente em mostrar que certos fatores de fato influenciam o tamanho das coalizões. Utilizando observações anuais dos gabinetes de 14 países da América Latina onde o partido dos incumbentes dispunha de menos de 50\% dos votos, Figueiredo et al. (\citeyear{figueiredo2012}) testaram algumas das hipóteses sobre a ocorrência de coalizões minoritárias, aquelas que não possuem maioria de apoio no congresso. Entre outros, eles mostram evidências de que vetos parciais e vetos difíceis de serem derrubado aumentam a probabilidade de surgirem coalizões minoritárias; por outro lado, assembleias fragmentadas e o efeito do ciclo eleitoral diminuiriam estas probabilidades. Já com análises \textit{time-series}, tanto Raile et al. (\citeyear{raile2010}) quanto Acosta e Polga-Hecimobich (\citeyear{acosta2011}) mostram que no Brasil e no Equador, respectivamente, o uso estratégico das emendas parlamentares pode compensar a cooperação dos membros da coalizão e evitar perdas de suporte legislativo ou mesmo a deserção de algum partido. Entretanto, outros estudos sugerem que, ao invés de cooperarem, presidentes podem usar de seus poderes legislativos para simplesmente contornar as assembleias: por exemplo, presidentes que contam com a prerrogativa de expedir decretos legislativos tendem a distribuir menos proporcionalmente ministérios entre seus parceiros de coalizão e a ter gabinetes menores e mais instáveis (AMORIM NETO, \citeyear{neto2006}; FIGUEIREDO et al., \citeyear{figueiredo2012}; MARTINEZ-GALLARDO, \citeyear{martinez2012}) \footnote{Qualificando esse efeito dos decretos legislativos, Negretto (\citeyear{negretto2004}) afirma que, onde existem, os presidentes só poderiam usá-los efetivamente quando controlam o membro pivotal do congresso, isto é, quando contam com uma base suficiente para impedir a aprovação de vetos - como foi o caso dos Presidentes Alfonsín e Menem na Argentina. Na ausência desta condição, expedir decretos funcionaria apenas como uma ferramenta de coordenação e delegação, reduzindo e transferindo os custos da tomada de decisões do congresso para o executivo}.
 
Enfim, apesar de estes estudos certamente ampliarem o nosso entendimento de como governos de coalizão funcionam, ainda não sabemos quais fatores explicam as variações entre eles. Coalizões podem ter diversos tamanhos e composições ideológicas diferentes, dependendo da decisão do presidente. Contudo, esta decisão, conforme argumentado, é parte de um jogo que envolve mais atores e um contexto que restringe o conjunto de estratégias disponíveis a cada um. Sobre isto, a teorização e as evidências sobre governos multipartidários na América Latina tendem a supervalorizar as experiências de alguns poucos países do Cone Sul, grosso modo, e não nos fornecem proposições específicas sobre a formação coalizões sobredimensionadas -- o que contrasta com a literatura sobre o parlamentarismo, onde o estudo deste tipo de coalizão já recebeu várias contribuições (Cf., VOLDEN e CARRUBA, \citeyear{volden2004}). Se governos de coalizão são podem surgir no presidencialismo e no parlamentarismo por razões semelhantes (Cf.CHEIBUB et al., \citeyear{cheibub2004}; CHEIBUB, \citeyear{cheibub2007}), falta ainda, portanto, testar se fatores comuns aos dois sistemas também são capazes de explicar a variação no tamanho das coalizões: o que, especificamente, um presidente ganha ao incluir mais partidos em seu gabinete, apesar dos custos de ter de dividir mais cargos e de barganhar com mais partidos? Qual é a influência que a relação entre executivo e legislativo desempenha na ocorrência desse fenômeno? E por que em alguns países estes tipos de coalizão não são formadas, como nos da América Central, e em outros, como Brasil e Chile, são formadas com frequência? No restante do artigo, procuro justamente contribuir para este debate através da análise dos determinantes da ocorrência de coalizões sobredimensionadas no presidencialismo latino-americano.

