\chapter{O tamanho das coalizões}
\label{chap:revisao}

Governos de coalizão são comuns no parlamentarismo. Segundo a explicação corrente, a razão disso estaria nos incentivos gerados pela dependência do executivo em relação ao legislativo. Quando o partido do primeiro-ministro não dispõe de maioria, o apoio de uma coalizão, mesmo que tácito, é necessário para evitar que o governo caia (LAVER e SHEPSLE, \citeyear{laver1996}; STR\O{}M, \citeyear{strom1990}). Desse modo, por não estar em questão as causas dessas coalizões, os estudos sobre o parlamentarismo puderam se voltar principalmente ao problema da variação entre elas: quantos partidos participam, quais deles e por quanto tempo (Cf., LAVER e SCHOFIELD \citeyear{laver1998}). Nos primeiros estudos sobre o tema no presidencialismo, ao contrário, a possibilidade mesma de surgirem governos de coalizão é que estava em causa (CHEIBUB, \citeyear{cheibub2002}; FOWERAKER, \citeyear{foweraker1998}; LINZ, \citeyear{linz1990}; MAINWARING, \citeyear{mainwaring1993}; STEPAN e SKACH, \citeyear{stepan1993}). No que segue, procuro argumentar que, justamente por isso, outros aspectos das coalizões governamentais no presidencialismo foram muito menos explorados. O tamanho delas é um exemplo.

\section{Tamanho das coalizões no parlamentarismo}

O que um partido ganha entrando numa coalizão? E quais fatores explicam por que algumas coalizões possuem mais partidos do que outras? A literatura sobre o parlamentarismo, mais antiga e extensa, nos fornece um bom ponto de partida para encontrar algumas dessas respostas. Grosso modo, três são as principais variáveis explicativas apontadas por ela: \textit{distribuição de cadeiras}, \textit{preferências ideológicas} e \textit{instituições}. As instituições porque estruturariam o processo de negociação e montagem de governos e, assim, delimitariam o número de coalizões que poderiam ser efetivamente formadas (STR\O{}M et al., \citeyear{strom1994}); já o número de cadeiras e a posição ideológica dos partidos, porque indicariam quais dessas coalizões seriam mais viáveis (AXELROD, \citeyear{axelrod1970}).

O marco inicial dessa literatura é bastante conhecido. Políticos procurariam usufruir de cargos públicos, que são bens escassos e de uso exclusivo, e, para tanto, formariam uma maioria na assembléia para controlar o governo. Dentre todas as coalizões que os permitiriam fazer isso, a ótima seria aquela que contaria com o mínimo de membros, a \textit{minimum winning coaltion} (MWC) (RIKER, (\citeyear{riker1962}). Na perspectiva dos políticos, isso seria vantajoso porque minimizaria o número de pessoas com quem eles teriam que dividir cargos, que são fixos no curto-prazo. Como todos enfrentam a mesma situação, a cooperação prospera e coalizões mínimas, mas capazes de satisfazer o critério de decisão majoritária, surgiriam. Embora não tenha sido o primeiro a formular esse modelo\footnote{Antes dele, Gamsom (\citeyear{gamson1961}) o formalizou de forma ampla; ele próprio, contudo, afirma que seu modelos inspira-se numa tradição alguns anos mais antiga, oriunda das primeiras contribuições à teoria dos jogos.}, Riker (\citeyear{riker1962}) o aplicou à situação concreta de formação de maiorias numa assembleia e derivou daí uma proposição bastante precisa sobre a coalizão que surgiria dada um distribuição de cadeiras\footnote{Cabe notar, do que fica implícito nesta proposição, que basta apenas a defecção de um membro da coalizão para que esta perca o controle do \textit{spoil}. Na prática, contudo, uma coalizão alternativa deve substituir a anterior para que este \textit{spoil} mude de mãos. Isto ocorre porque, na maioria dos países parlamentaristas, no caso de ausência de uma maioria alternativa, um governo \textit{caretaker} assume, geralmente composto por membros do governo anterior e comprometida em não alterar o \textit{status quo} (LAVER e SCHEPSLE, \citeyear{laver1996}, p. 47-8)}.

A simplicidade desse modelo \textit{office-seeking} \footnote{Para uma discussão sobre outros modelos \textit{office-based}, ver Martin e Stevenson (\citeyear{martin2001}) e Crombez (\citeyear{crombez1996}).}, entretanto, foi justamente o principal objeto de crítica da literatura subsequente. Entre outros, Luebbert (\citeyear{luebbert1986}) e Str\o{}m (\citeyear{strom1990}) mostraram que coalizões minoritárias não apenas surgiam com frequência, mas também que estas eram tão estáveis e poderiam durar por tanto tempo quanto suas correlatas majoritárias. Deste modo, Riker criou um \textit{puzzle} enorme: por que a oposição permite a sobrevivência de gabinetes minoritários? Ou bem existiria alguma compensação (\textit{side-payment}) aos partidos de fora do governo, ou então a utilidade destes seria derivada de outros locais.
 
Na esteira desse problema, emergiu a ideia de que não apenas cargos importavam, mas também a ideologia. A não ocupação de cargos poderia ser compensada, ou mesmo substituída, pela implementação de uma agenda legislativa da preferência dos partidos. Uma primeira explicação para isso seria a de que os membros de uma coalizão, por exemplo, buscariam diminuir os conflitos entre eles. Como nem sempre numa barganha dois partidos conseguem firmar acordo - seja por desconfiança mútua, falta de informações sobre as consequências do acordo, entre outros -, dois partidos quaisquer em coalizão poderiam incluir nela todos aqueles que estivessem situados entre eles numa escala ideológica unidimensional para amenizar esse problema, o que poderia gerar coalizões maiores do que as \textit{MWC} (AXELROD, 1970)\footnote{A explicação de Axelrod (1970) é a seguinte: se o potencial de conflito entre dois partidos aumenta quanto maior for a distância entre eles numa dimensão qualquer, e se o potencial de conflito numa coalizão é a média dos potenciais de conflitos entre todos os pares não repetidos de partidos, coalizões com menor variação nas preferências dos seus membros, isto é, coalizões mais homogêneas, gerariam maiores ganhos de estabilidade.}. Com isso, a manutenção da coalizão se tornaria mais fácil porque o novo integrante assumiria, de forma prática, o papel de mediano da coalizão, evitando que este tivesse que ser negociado. No caso das coalizões mínimas, a explicação poderia ser ainda mais direta: um partido de oposição pode simplesmente se beneficiar das políticas do governo e, por isso, não ter incentivos para derrubá-lo. Entre outros, a principal vantagem de fazer isso seria a de se poupar de desgastes com a administração pública, particularmente quando medidas impopulares precisam ser implementadas. Antecipando esse comportamento, o partido encarregado de formar uma coalizão, o \textit{formateur}, pode estrategicamente manter partidos ideologicamente próximos fora do governo e reter inteiramente o executivo. Tacitamente, contudo, uma coalizão existiria (STR\O{}M, \citeyear{strom1990}).

Distribuição de cadeiras e de preferências, portanto, podem influenciar o tamanho de uma coalizão, mas não fazem isso num vácuo institucional. Um conjunto de regras estrutura o processo de formação de gabinetes no parlamentarismo, determinando as estratégias de cada partido. Enquanto que, em tese, uma assembléia fragmentada oferece diversas possibilidades de coalizão, certas regras podem alterar drasticamente esse número (STR\O{}M et al., \citeyear{strom1994}). Por exemplo, dependendo de como o partido \textit{formateur} é escolhido, outros partidos podem recusar propostas de integrar a coalizão -- desde que tenham chances de se tornar o próximo \textit{formateur} caso as negociações em curso falhem, o que justamente lhes dá incentivos para não cooperar (AUSTEN-SMITH e BANKS, \citeyear{austen1988}). A divisão do governo em jurisdições específicas comandadas por ministérios, por sua vez, também pode dificultar a acomodação de muitos partidos num gabinete, pois torna difícil o monitoramento de cada ministério individualmente e aumenta as chances de que surjam conflitos intragovernamentais (LAVER e SHEPSLE, \citeyear{laver1996}). Deste modo, a consideração das regras que delimitam os cursos de ação disponíveis aos partidos seriam necessárias para tornar mais precisos os modelos sobre formação de gabinetes multipartidários.

Em suma, de acordo com essa literatura discutida, o contexto institucional e o número e as preferências dos partidos explicariam boa parte da variação no tamanho das coalizões formadas no parlamentarismo. O número de cadeiras seria um bom indicador da necessidade do partido \textit{formateur} buscar ou não uma coalizão e do número de partidos em coalizão necessários para se obter maioria. Preferências ideológicas, por outro lado, apontariam para como os partidos governariam e, portanto, seriam bons indicadores da probabilidade de que dois partidos quaisquer viessem a cooperar. Dependendo da posição e do número de cadeiras dos demais, coalizões mínimas ou sobredimensionadas podem surgir. Por fim, as regras que estruturam o processo de formação de coalizões e exercício do governo também deveriam ser levadas em conta para lidar com variações entre países e contextos, bem como para entender como a sequência de movimentos e as opções disponíveis a cada partido levam a equilíbrios diferentes dos que modelos \textit{office} ou \textit{policy-seeking} isoladamente indicariam. 


\section{Tamanho das coalizões no presidencialismo}

Ao contrário da literatura sobre o parlamentarismo, os estudos sobre sistemas presidencialistas até pouco tempo afirmavam que coalizões governamentais não deveriam surgir. Dois argumentos principais sustentariam essa conclusão. O primeiro é o de que, legitimados pela maioria dos eleitores, presidentes não dependeriam do apoio do legislativo para manter seus cargos. Essa seria a essência da dinâmica \textit{winner-takes-all} inerente ao presidencialismo (LINZ, \citeyear{linz1990}; LINZ e VALENZUELA, \citeyear{linz1994}; RIGGS, \citeyear{riggs1988}). O segundo argumento, desenvolvido por outra geração de comparativistas, é o de que certos aspectos institucionais comuns ao presidencialismo latino-americano dificultariam o surgimento de coalizões (MAINWARING, \citeyear{mainwaring1993}; SHUGART e CAREY, \citeyear{shugart1992}; STEPAN e SKACH, \citeyear{stepan1993}). Representação proporcional para eleições legislativas tornaria bastante provável que o partido do presidente não contasse com apoio legislativo de uma maioria. Sistemas eleitorais inclusivos, por outro lado, gerariam incentivos para a personalização e a regionalização das campanhas eleitorais, o que em último caso enfraqueceria a coesão dos partidos políticos e dificultaria a formação e a manutenção de coalizões. Deste modo, os presidentes da região não teriam incentivos para cooperar com as assembleias, preferindo antes contorná-las ou usar incentivos seletivos para formar coalizões legislativas \textit{ad hoc} (COX e MORGENSTERN, \citeyear{cox2001}; JONES, \citeyear{jones1995})\footnote{Reconstituir este debate está fora do escopo deste artigo. Boas revisões, entretanto, podem ser encontradas em Cheibub (\citeyear{cheibub2007}), Power (\citeyear{power2010}) e Negretto (\citeyear{negretto2006}).}.

Como essa premissa de ausência de cooperação entre executivo e legislativo colocava em causa a própria estabilidade democrática e a governabilidade dos países da América Latina, os primeiros estudos sobre o presidencialismo buscaram invariavelmente explorar as condições favoráveis à formação de governos multipartidários ao invés das diferenças entre estes. Passados alguns anos do início deste debate, hoje sabemos que coalizões governamentais são comuns e que auxiliam na aprovação da agenda presidencial, na diminuição de conflitos intergovernamentais e na sustentação de presidentes que enfrentam crises econômicas ou protestos populares (ÁLVAREZ e MARSTEINTREDET, \citeyear{alvarez2010}; CHEIBUB, \citeyear{cheibub2007}; CHEIBUB et al., \citeyear{cheibub2004}; HOCHSTETLER, \citeyear{hochstetler2006}; NEGRETTO, \citeyear{negretto2006}; PÉREZ-LIÑÁN, \citeyear{perez2007})\footnote{Na Colômbia e na Venezuela, assembleias constituintes foram convocadas e, durante seus trabalhos, dissolveram o congresso. Ainda que não tenha sido por intervenção militar, como o fracassado auto-golpe de Fujimori, no Peru, estas dissoluções não eram diretamente previstas pelas respectivas constituições –- ao contrário da uruguaia, que prevê tal prerrogativa, apesar de que, na prática, ela não seja usada (SHUGART e CAREY, \citeyear{shugart1992}, p. 127)}. Recentemente, também surgiram estudos mostrando que coalizões de diferentes tamanhos trazem consequências distintas, como promover a autonomia de agências regulatórias quando são estabelecidas, aumentar o grau de obstrução legislativa à agenda do presidente e facilitar a implementação de reformas estruturais (ALTMAN e CASTIGLIONI, \citeyear{altman2008}; HIROI e RENNÓ, \citeyear{hiroi2014}; MELO e PEREIRA, \citeyear{melo2013}). Mas, apesar desses avanços, à exceção de uns poucos estudos que analisam a composição e a estabilidade de gabinetes presidencialistas comparativamente (e. g., AMORIM NETO, \citeyear{neto2006}; FIGUEIREDO et al., \citeyear{figueiredo2012}; MARTINEZ-GALLARDO, \citeyear{martinez2012}), ainda sabemos pouco sobre o que explica essas diferenças no tamanho das coalizões que são formadas no Continente em primeiro lugar.

Em parte, o tamanho das coalizões poderia ser considerado função exclusiva da decisão dos presidentes, já que eles possuem a prerrogativa de nomear os ministros no presidencialismo. Contudo, esse dificilmente é o caso. A montagem de um gabinete multipartidário envolve a interação estratégica de diversos atores com preferências, poder de barganha e prerrogativas institucionais diferentes. Em decorrência disso, segundo Cheibub (\citeyear{cheibub2007}), a formação de uma coalizão é um jogo no qual os presidentes fazem propostas para os potenciais parceiros considerando os custos e a utilidade de tê-los cooperando e a probabilidade de eles aceitarem essas propostas. Se todos os jogadores procurarem implementar políticas públicas e obter cargos, coalizões surgiriam exceto quando as preferências ideológicas do presidente fossem tão extremas que nenhum partido obtivesse vantagem ao integrar o gabinete; ou quando o presidente estivesse centralmente localizado no espectro ideológico e fosse naturalmente o ponto de convergência da maioria. Em outras palavras, ainda que sejam eleitos separadamente e muitas vezes contem com amplos poderes legislativos, presidentes não escolhem arbitrariamente suas coalizões, já que as preferências dos demais partidos e o contexto no qual as negociações ocorrem produzem incentivos que determinam parcialmente os resultados do processo.

Da evidência disponível sobre governos de coalizão no presidencialismo, a maioria é consistente em mostrar que certos fatores de fato influenciam o tamanho das coalizões. Utilizando observações anuais dos gabinetes de 14 países da América Latina onde o partido dos incumbentes dispunha de menos de 50\% dos votos, Figueiredo et al. (\citeyear{figueiredo2012}) testaram algumas das hipóteses sobre a ocorrência de coalizões minoritárias, aquelas que não possuem maioria de apoio no congresso. Entre outros, eles mostram evidências de que vetos parciais e vetos difíceis de serem derrubado aumentam a probabilidade de surgirem coalizões minoritárias; por outro lado, assembleias fragmentadas e o efeito do ciclo eleitoral diminuiriam estas probabilidades. Já com análises \textit{time-series}, tanto Raile et al. (\citeyear{raile2010}) quanto Acosta e Polga-Hecimobich (\citeyear{acosta2011}) mostram que no Brasil e no Equador, respectivamente, o uso estratégico das emendas parlamentares pode compensar a cooperação dos membros da coalizão e evitar perdas de suporte legislativo ou mesmo a deserção de algum partido. Entretanto, outros estudos sugerem que, ao invés de cooperarem, presidentes podem usar de seus poderes legislativos para simplesmente contornar as assembleias: por exemplo, presidentes que contam com a prerrogativa de expedir decretos legislativos tendem a distribuir menos proporcionalmente ministérios entre seus parceiros de coalizão e a ter gabinetes menores e mais instáveis (AMORIM NETO, \citeyear{neto2006}; FIGUEIREDO et al., \citeyear{figueiredo2012}; MARTINEZ-GALLARDO, \citeyear{martinez2012}) \footnote{Qualificando esse efeito dos decretos legislativos, Negretto (\citeyear{negretto2004}) afirma que, onde existem, os presidentes só poderiam usá-los efetivamente quando controlam o membro pivotal do congresso, isto é, quando contam com uma base suficiente para impedir a aprovação de vetos - como foi o caso dos Presidentes Alfonsín e Menem na Argentina. Na ausência desta condição, expedir decretos funcionaria apenas como uma ferramenta de coordenação e delegação, reduzindo e transferindo os custos da tomada de decisões do congresso para o executivo}.
 
Enfim, apesar de estes estudos certamente ampliarem o nosso entendimento de como governos de coalizão funcionam, ainda não sabemos quais fatores explicam as variações entre eles. Coalizões podem ter diversos tamanhos e composições ideológicas diferentes, dependendo da decisão do presidente. Contudo, esta decisão, conforme argumentado, é parte de um jogo que envolve mais atores e um contexto que restringe o conjunto de estratégias disponíveis a cada um. Sobre isto, a teorização e as evidências sobre governos multipartidários na América Latina tendem a supervalorizar as experiências de alguns poucos países do Cone Sul, grosso modo, e não nos fornecem proposições específicas sobre a formação coalizões sobredimensionadas -- o que contrasta com a literatura sobre o parlamentarismo, onde o estudo deste tipo de coalizão já recebeu várias contribuições (Cf., VOLDEN e CARRUBA, \citeyear{volden2004}). Se governos de coalizão são podem surgir no presidencialismo e no parlamentarismo por razões semelhantes (Cf.CHEIBUB et al., \citeyear{cheibub2004}; CHEIBUB, \citeyear{cheibub2007}), falta ainda, portanto, testar se fatores comuns aos dois sistemas também são capazes de explicar a variação no tamanho das coalizões: o que, especificamente, um presidente ganha ao incluir mais partidos em seu gabinete, apesar dos custos de ter de dividir mais cargos e de barganhar com mais partidos? Qual é a influência que a relação entre executivo e legislativo desempenha na ocorrência desse fenômeno? E por que em alguns países estes tipos de coalizão não são formadas, como nos da América Central, e em outros, como Brasil e Chile, são formadas com frequência? No restante do artigo, procuro justamente contribuir para este debate através da análise dos determinantes da ocorrência de coalizões sobredimensionadas no presidencialismo latino-americano.

