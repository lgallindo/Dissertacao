\chapter{Discussão}
\label{chap:discussao}

A análise anterior buscou esclarecer o que explica o surgimento de coalizões governamentais no presidencialismo latino-americano. Ao partilharem o poder executivo, presidentes parecem ter maiores incentivos para formar coalizões governamentais sobredimensionadas quando a fragmentação partidária é elevada, quando a polarização ideológica é baixa e quando o congresso é capaz de vetar mudanças no \textit{status quo} e atuar de maneira efetiva. Nestes casos, coalizões grandes podem ser utilizadas porque diminuem a capacidade dos partidos de oposição de usar o legislativo contra os interesses do executivo e também porque reduzem a dependência do presidente em relação a um ou mais partidos. O que não parece encontrar suporte na literatura, por outro lado, é o efeito positivo que maiores poderes legislativos exercem na probabilidade de um presidente formar um gabinete grande. A despeito disso, a performance geral dos principais modelos estimados é bastante satisfatória. Mas como eles nos ajudam a entender casos concretos de coalizões sobredimensionadas? Aqui, contextualizo estes resultados e discuto alguns dos casos desviantes.

Dos países com ocorrências de gabinetes \textit{surplus} na amostra, o Brasil é o que mais os teve desde a redemocratização (22 anos entre 1985 e 2012, ou 81\% do período). Como se sabe, o presidente brasileiro possui uma ampla gama de poderes, como o de conduzir o processo orçamentário e o de expedir Medidas Provisórias, o que lhe possibilita optar por diversas estratégias de governabilidade (RAILE et al., \citeyear{raile2010}). Por sua vez, o Congresso Nacional também conta com importantes prerrogativas, como a de criar Comissões Parlamentares de Inquérito (CPI) e a de fiscalizar as contas públicas, além de um grande número de Comissões Permanentes que, com cada deputado podendo integrar apenas uma, facilitam a especialização (MELO e PEREIRA, \citeyear{melo2013}; PEREIRA e MULLER, \citeyear{pereira2000}). Aliada à centralização do processo decisório nos líderes e nos cargos de direção, que facilita a coordenação da atividade legislativa (FIGUEIREDO e LIMONGI, \citeyear{figueiredo1999}), estas prerrogativas dão ao Congresso meios efetivos para barrar o poder executivo e, deste modo, o incentivar a cooperar. 

Com efeito, um argumento bastante recorrente na opinião pública para explicar a existência de coalizões grandes e ausência de reformas no Brasil é o de que elas serviriam para impedir a criação de CPI's que possam prejudicar o governo\footnote{A entrevista do então Ministro das Relações Institucionais Walfrido Mares Guia à Folha, em 2007, é um bom exemplo disso (FOLHA, \citeyear{folha2007})}. O caso mais emblemático disso certamente é o da CPI dos correios, aprovada com o apoio, entre outros, de parlamentares do Partido do Movimento Democrático Brasileiro (PMDB), que na sequência passou a integrar o governo. Isto é exemplificado por dois dos cinco maiores casos de \textit{overpreddicting} do modelo 3, os do governo Lula em 2002 e em 2003: a probabilidade prevista pelo modelo de que coalizões \textit{surplus} seriam formadas nestes casos eram de 96\% e 78\%, mas os eventos previstos não ocorreram. Passados o ano de maior impacto da crise envolvendo o escândalo do mensalão e a eleição de 2006, de outro modo, todos os governos subsequentes do Partido dos Trabalhadores (PT) passaram a manter coalizões amplas\footnote{O modelo 3 prediz corretamente estes casos, atribuindo probabilidades de ocorrência de coalizões sobredimensionadas maiores que 95\%.}, provavelmente por causa do aprendizado com a experiência passada.

Algo semelhante ao que ocorreu no Brasil em 2002-3 também aconteceu na Bolívia. Com um legislativo bicameral, um sistema eleitoral misto e elevada polarização no congresso, agravada com o crescente sucesso eleitoral do Movimento al Socialismo (MAS), nenhum presidente boliviano governou sem recorrer à formação de uma coalizão governamental entre 1994 e 2002, quando se inicia um período de instabilidade que só termina com a eleição Evo Morales, em 2006 (DUNKERLEY, \citeyear{dunkerley2007}). Mas, se antes de 94 coalizões sobredimensionadas eram infrequentes, após a reforma eleitoral elas tornam-se predominantes. Neste cenário, o ano da posse do segundo governo de Gonzalo de Lozada seria apenas outro de coalizão \textit{surplus} não fosse a profunda crise econômica e social que tomou conta do país e, em última instância, o forçou a renunciar depois do evento conhecido como "guerra do gás". Mantidas quase as mesmas condições de 2002, o modelo 3 prevê que o governo seguinte, do independete Carlos Mesa, teria mais de 50\% de probabilidade de também ser uma coalizão sobredimensionada. Contudo, tanto Mesa quanto seu sucessor, Rodríguez Veltez, preferiram lidar com a crise através de gabinetes apartidários, com \textit{experts} ocupando os cargos ministeriais e contando com o suporte geral dos partidos e da opinião pública (BREUER, \citeyear{breuer2008}, p. 15-6). Como o exemplo do Brasil também sugere, portanto, momentos de transição e crises políticas talvez ajudem a explicar interrupções em sequências de coalizões sobredimensionadas.

Dois outros casos, enfim, exemplificam a variação \textit{entre} coalizões de tipo sobredimensionado. O primeiro deles é o do Chile, onde rotineiramente apenas um pequeno partido excedente integra a coalizão governista -- diferentemente, portanto, do que ocorre no Brasil, na Bolívia e na Colômbia. O efeito da transição democrática na consolidação de duas grandes coalizões, a \textit{Concertación} e a \textit{Alianza por Chile}, é essencial na configuração desse fenômeno. O plebiscito votado em 1988 sobre a continuidade ou não do regime sob Pinochet aglutinou o espectro político em dois polos principais, que foi posteriormente transplantado para o governo. Aleman e Saiegh (\citeyear{aleman2007}) mostram que essas alianças são bem-estruturadas e homogêneas, induzem a coordenação legislativa de forma efetiva e permitem dizer que, na prática, o sistema partidário chileno funciona como um bipartidarismo. Deste modo, embora os partidos no gabinete não variem entre governos de cada aliança, a distribuição de cadeiras é a principal variável que explica porque algumas coalizões são sobredimensionadas e outras, não.

Por fim, o segundo caso que merece comentário é o da coalizão sobredimensionada existente República Dominicana entre 2006 e 2009. Na América Central como um todo, governos de coalizão são raros, e a maioria dos sistemas partidários da região possuem poucos partidos e estes geralmente controlam a formação das listas partidárias, o que, em tese, lhes proporcionaria maior capacidade de impor disciplina no comportamento parlamentar (Cf. SHUGART e CAREY, \citeyear{shugart1992}). Além disso, os sucessivos conflitos civis e o histórico de intervenções americanas são apontados como causas da extrema polarização ideológica que, por sua vez, dificulta a cooperação interpartidária nestes países (SCOTT e MARSHALL, \citeyear{scott1998}). Distribuir ministérios em troca de apoio, deste modo, não é um meio usualmente empregado na região, e a República Dominicana não é uma exceção quanto a isto. Ao contrário, a fórmula principal de governabilidade no país desde a redemocratização, em 1978, tem sido a distribuição de cargos de direção no legislativo e nas administrações locais e incentivos seletivos -- não raro dinheiro -- em troca de apoio (MARSTEINTREDET, \citeyear{mars2008}). A exceção durante o segundo mandato do Presidente Leonel Fernández, do \textit{Partido de la Liberación Dominicana} (PLD), que formou uma coalizão sobredimensionada em 2006, só ocorreu porque seu partido contava com 54\% das cadeiras na câmara baixa, mas incorporou um membro do \textit{Partido Reformista Social Cristiano} (PRSC) no \textit{Ministério de relaciones Exteriores}. De acordo com Benito Sánchez (\citeyear{benito2010}, p. 755), este movimento não visava apenas trazer os 12\% de cadeiras do PRSC para o lado governista, mas, precisamente, esvaziar o partido após a morte do seu principal líder, o ex-Presidente Joaquín Balaguer. Por esta razão, o caso dominicano é verdadeiramente desviante e não se encaixa no quadro geral esboçado.



